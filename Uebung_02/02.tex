\newcommand{\authorinfotitle}{Vanessa Closius, Jonas Tietz, Tronje Krabbe}
\newcommand{\authorinfo}{Vanessa Closius, Jonas Tietz, Tronje Krabbe}
\newcommand{\titleinfo}{MMS}
\newcommand{\qed}{\square}

\documentclass [a4paper,11pt]{article}
\usepackage[german,ngerman]{babel}
\usepackage[utf8]{inputenc}
\usepackage[T1]{fontenc}
\usepackage{lmodern}
\usepackage{amssymb}
\usepackage{mathtools}
\usepackage{amsmath}
\usepackage{enumerate}
\usepackage{breqn}
\usepackage{fancyhdr}
\usepackage{multicol}

\author{\authorinfotitle}
\title{\titleinfo}
\date{\today}

\pagestyle{fancy}
\fancyhf{}
\fancyhead[R]{\authorinfo}
\fancyhead[L]{MMS Hausaufgaben}
\fancyfoot[C]{\thepage}

\begin{document}
	\maketitle
	\begin{enumerate}
		% Aufgabe 1
		\item[\textbf{1.}]
		Für Aufgabe 1 haben wir die folgenden Formeln aus der Vorlesung verwendet.
		\begin{equation}
		\int_{-\pi}^\pi(\sin(k\omega_0x)\cos(m\omega_0x))dx = 0 \text{, für } k,m\in \mathbb{Z}
		\end{equation}
		\begin{equation}
		\int_{-\pi}^{\pi}(\sin(mx)\sin(nx))dx = \delta_{m,n}\pi
		\end{equation}
		\begin{equation}
		\int_{-\pi}^{\pi}(\cos(mx)\cos(nx))dx = \delta_{m,n}\pi
		\end{equation}
		Des weiteren noch:
		\begin{equation}
		\int \cos(kt)dt = \frac{\sin(kt)}{k}
		\end{equation}
		\begin{equation}
		\cos(t+\pi) = -\cos(t)
		\end{equation}
		\begin{equation}
		\sin(-t) = -\sin(t)
		\end{equation}
		\begin{equation}
		\cos(-t) = \cos(t)
		\end{equation}
		\begin{enumerate}
			\item[a)] $f(t) = \sin(t)$
			$$\alpha_k = \frac{1}{\pi} \int_{-\pi}^{\pi}(\sin(t)\cos(kt))dt \underbrace{=}_{(1)} 0$$

			$$\beta_k = \frac{1}{\pi} \int_{-\pi}^{\pi}(\sin(t)\sin(kt))dt$$
			Hier gilt wegen (2):
			$$\beta_1 = \frac{1}{\pi} \cdot \pi = 1$$
			$$\beta_k = 0 \text{, für } k > 1$$

			\item[b)] $f(t) = \cos(2t)$

			$$\alpha_k = \frac{1}{\pi}\int_{-\pi}^{\pi}(\cos(2t)\cos(kt))dt$$
			Hier gilt wegen (3):
			$$\alpha_2 = \frac{1}{\pi} \cdot \pi = 1$$
			$$\alpha_k = 0 \text{, für } k\neq2$$

			$$\beta_k = \frac{1}{\pi}\int_{-\pi}^{\pi}(\cos(2t)\sin(kt))dt \underbrace{=}_{(1)} 0$$

			\item[c)] $f(t) = 1$

			\begin{dmath*}
				\alpha_k = \frac{1}{\pi}\int_{-\pi}^{\pi}(\cos(kt))dt
				= \frac{1}{\pi}\left[ \frac{\sin(kt)}{k}\right]_{-\pi}^{\pi}
				= \frac{1}{\pi}\left(\frac{\sin(k\pi)}{k}-\frac{\sin(-k\pi)}{k} \right) \\
				\underbrace{=}_{(6)} \frac{1}{\pi}\left(\frac{\sin(k\pi)}{k}+\frac{\sin(k\pi)}{k} \right)
				= \frac{1}{\pi}\left(2\frac{\sin(k\pi)}{k} \right)
				= \frac{2\sin(k\pi)}{k\pi}
			\end{dmath*}

			\begin{dmath*}
				\beta_k = \frac{1}{\pi}\int_{-\pi}^{\pi}(\sin(kt))dt
				= \frac{1}{\pi}\left[ \frac{\cos(kt)}{k}\right]_{-\pi}^{\pi}
				= \frac{1}{\pi}\left(\frac{\cos(k\pi)}{k}-\frac{\cos(-k\pi)}{k} \right) \\
				\underbrace{=}_{(7)} \frac{1}{\pi}\left(\frac{\cos(k\pi)}{k}-\frac{\cos(k\pi)}{k} \right)
				= 0
			\end{dmath*}

			\item[d)] $f(t) = \cos(5t+\pi)+3\sin(9t)$
			\begin{dmath*}
				\alpha_k = \frac{1}{\pi} \int_{-\pi}^{\pi}((\cos(5t+\pi)+3\sin(9t))\cos(kt))dt
				= \frac{1}{\pi} \int_{-\pi}^{\pi}(\cos(5t+\pi)\cos(kt)+3\sin(9t)\cos(kt))dt
				= \frac{1}{\pi}\left( \int_{-\pi}^{\pi}(\cos(5t+\pi)\cos(kt))dt+\int_{-\pi}^{\pi}(3\sin(9t)\cos(kt))dt \right) \\
				\underbrace{=}_{(1)} \frac{1}{\pi} \int_{-\pi}^{\pi}(\cos(5t+\pi)\cos(kt))dt \\
				\underbrace{=}_{(5)} -\frac{1}{\pi} \int_{-\pi}^{\pi}(\cos(5t)\cos(kt))dt
			\end{dmath*}
			Und dann gilt aufgrund von (3):
			$$\alpha_5 = -\frac{1}{\pi} \cdot \pi = -1$$
			$$\alpha_k = 0 \text{, für } k \neq 5$$

			\begin{dmath*}
				\beta_k = \frac{1}{\pi} \int_{-\pi}^{\pi}((\cos(5t+\pi)+3\sin(9t))\sin(kt))dt
				= \frac{1}{\pi} \int_{-\pi}^{\pi}(\cos(5t+\pi)\sin(kt)+3\sin(9t)\sin(kt))dt
				= \frac{1}{\pi}\left( \int_{-\pi}^{\pi}(\cos(5t+\pi)\sin(kt))dt+\int_{-\pi}^{\pi}(3\sin(9t)\sin(kt))dt \right) \\
				\underbrace{=}_{(5)} \frac{1}{\pi}\left( -\int_{-\pi}^{\pi}(\cos(5t)\sin(kt))dt+\int_{-\pi}^{\pi}(3\sin(9t)\sin(kt))dt \right) \\
				\underbrace{=}_{(1)} \frac{1}{\pi} \int_{-\pi}^{\pi}(3\sin(9t)\sin(kt))dt
				= \frac{3}{\pi} \int_{-\pi}^{\pi}(\sin(9t)\sin(kt))dt
			\end{dmath*}
			Und dann gilt aufgrund von (2):
			$$\beta_9 = \frac{3}{\pi} \cdot \pi = 3$$
			$$\beta_k = 0 \text{, für } k \neq 9$$
		\end{enumerate}
	\end{enumerate}
\end{document}
