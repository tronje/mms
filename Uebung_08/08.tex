\newcommand{\authorinfotitle}{Vanessa Closius, Jonas Tietz, Tronje Krabbe}
\newcommand{\authorinfo}{Vanessa Closius, Jonas Tietz, Tronje Krabbe}
\newcommand{\titleinfo}{MMS}
\newcommand{\qed}{\square}

\documentclass[a4paper,11pt]{article}
%\usepackage[german,ngerman]{babel}
\usepackage[utf8]{inputenc}
\usepackage[T1]{fontenc}
\usepackage{lmodern}
\usepackage{amssymb}
\usepackage{mathtools}
\usepackage{amsmath}
\usepackage{enumerate}
\usepackage{breqn}
\usepackage{fancyhdr}
\usepackage{multicol}
\usepackage{tikz}
\usepackage{trfsigns}

\author{\authorinfotitle}
\title{\titleinfo}
\date{\today}

\pagestyle{fancy}
\fancyhf{}
\fancyhead[R]{\authorinfo}
\fancyhead[L]{MMS Hausaufgaben}
\fancyfoot[C]{\thepage}
\allowdisplaybreaks
\begin{document}
	\maketitle
	\begin{enumerate}
		% Aufgabe 1
		\item[\textbf{1.}]
			\begin{enumerate}
				\item[b)]
				Die Bandpasscharakteristik der Gabor-Transformation lässt sich leicht aus ihrer Formel $ e^{j\omega t}\cdot G(t,\sigma)$ herleiten. $e^{j\omega t}$ lässt sich nach der eulerschen Formel als
				$e^{j\omega t} = \cos(\omega t) + j\sin(\omega t)$ darstellen.
				Im Frequenzbereich wird daraus sowohl ein gerades als auch ein ungerades Impulspaar. Aus der Multiplikation wird dann eine Faltung. Dadurch bekommt man im Frequenzbereich diese Formel: $(\frac{1}{2}(\delta(t+\omega)+\delta(t-\omega)) + \frac{1}{2}(\delta(t+\omega)-\delta(t-\omega)))\otimes F\lbrace G(t,\sigma) \rbrace$
				
				Laut den Replikationstheorem wird die Gauß'sche Dichtefunktion auf die Dirac-Stöße repliziert. Dadurch entsteht ein Bandpassfilter, dessen Breite sich mit der Standardabweichung $\sigma$ der Gauß'sche Dichtefunktion einstellen lässt und dessen \"Bandmitte\" der Frequenz $\omega$. 
			\end{enumerate}
	\end{enumerate}
\end{document}
