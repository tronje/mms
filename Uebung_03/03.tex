\newcommand{\authorinfotitle}{Vanessa Closius, Jonas Tietz, Tronje Krabbe}
\newcommand{\authorinfo}{Vanessa Closius, Jonas Tietz, Tronje Krabbe}
\newcommand{\titleinfo}{MMS}
\newcommand{\qed}{\square}

\documentclass[a4paper,11pt]{article}
\usepackage[german,ngerman]{babel}
\usepackage[utf8]{inputenc}
\usepackage[T1]{fontenc}
\usepackage{lmodern}
\usepackage{amssymb}
\usepackage{mathtools}
\usepackage{amsmath}
\usepackage{enumerate}
\usepackage{breqn}
\usepackage{fancyhdr}
\usepackage{multicol}
\usepackage{tikz}

\author{\authorinfotitle}
\title{\titleinfo}
\date{\today}

\pagestyle{fancy}
\fancyhf{}
\fancyhead[R]{\authorinfo}
\fancyhead[L]{MMS Hausaufgaben}
\fancyfoot[C]{\thepage}

\begin{document}
	\maketitle
	\begin{enumerate}
		% Aufgabe 1
		\item[\textbf{1.}]
		\begin{enumerate}
			\item[\textbf{1)}]
			\begin{enumerate}
			% b)
			\item[\textbf{a)}] $z_1 = 1+j\sqrt{3}$ $z_2=1-j$
			\begin{dmath*}
			z_1+z_2 = 1+1+j\sqrt{3}-j = 2+j(\sqrt{3}-1)
			\end{dmath*}
			\begin{dmath*}
			z_1-z_2 = 1-1+j\sqrt{3}+j = j(\sqrt{3}+1)
			\end{dmath*}
			\begin{dmath*}
			z_1 \cdot z_2 = (1+j\sqrt{3})(1-j)
			              = 1-j+j\sqrt{3}+\sqrt{3}
			              = (1+\sqrt{3})+j(\sqrt{3}-1)
			\end{dmath*}
			\begin{dmath*}
			z_1 / z_2 = \frac{(1+j\sqrt{3})}{(1-j)}
			          = \frac{1+\sqrt{3}}{1+1}+j\frac{\sqrt{3}-1}{1+1}
			          = \frac{1+\sqrt{3}}{2}+j\frac{\sqrt{3}-1}{2}
			\end{dmath*}
			\begin{dmath*}
			z_1^* \cdot z_2 = (1-j\sqrt{3})(1-j)
			              = 1-j-j\sqrt{3}-sqrt{3}
			              = (1-\sqrt{3})-j(\sqrt{3}+1)
			\end{dmath*}
			\begin{dmath*}
			z_1 / z_2^* = \frac{(1+j\sqrt{3})}{(1+j)}
			            = \frac{1+\sqrt{3}}{1+1}+j\frac{\sqrt{3}-1}{1+1}
			            = \frac{1+\sqrt{3}}{2}+j\frac{\sqrt{3}-1}{2}
			\end{dmath*}
			
			%b)
			\item[\textbf{b)}] $z_1 = 2+3j$ $z_2=3-5j$
			
			\begin{align*}
			z_1+z_2 &= 2+3+3j-5j \\
					&= 5-2j \\\\
			z_1-z_2 &= 2-3+3j+5j \\
					&= -1+8j \\\\
			z_1 \cdot z_2 &= (2+3j)(3-5j) \\
			              &= 6-10j+9j+15 \\
			              &= 21-j \\\\
			z_1 / z_2 &= \frac{(2+3j)}{(3-5j)} \\
			          &= \frac{6-15}{9+25}+j\frac{9+10}{9+25} \\
			          &= \frac{-9}{34}+j\frac{19}{34} \\\\
			z_1^* \cdot z_2 &= (2-3j)(3-5j) \\
			              	&= 6-10j-9j-15 \\
			              	&= -9-19j \\\\
			z_1 / z_2^* &= \frac{2+3j}{3+5j}\\
						&= \frac{6+15}{9+25} + j\frac{9-10}{9+25} \\
						&= \frac{21}{34}-k\frac{1}{34}
			\end{align*}
		\end{enumerate}
		\item[\textbf{2)}]
		$e^{i\theta}$ stellt einen Vektor in der komplexen Zahlenebene da, der um einen Winkel $\theta$ um den Einheitskreis rotiert worden ist. Mit der eulerschen Formel $e^{i\theta} = \cos(\theta)+i\sin(theta)$ bekommt man die kartesischen Koordinaten des rotierten Vektors. Da $\pi$ genau eine halbe Rotation um den Einheitskreis ist bekommt man $e^{i\theta} = -1$. Dies kann man dann noch umformen um Eulers Identität $e^{i\theta} +1 = 0$ zu erhalten.
		
		% Copy paste from http://www.texample.net/tikz/examples/unit-circle/
		\begin{tikzpicture}[scale=3.5,cap=round,>=latex]
        % draw the coordinates
        \draw[->] (-1.5cm,0cm) -- (1.5cm,0cm) node[right,fill=white] {$Re(z)$};
        \draw[->] (0cm,-1.5cm) -- (0cm,1.5cm) node[above,fill=white] {$Im(z)$};

        % draw the unit circle
        \draw[thick] (0cm,0cm) circle(1cm);

        \foreach \x in {0,30,...,360} {
                % lines from center to point
                %\draw[gray] (0cm,0cm) -- (\x:1cm);
                % dots at each point
                \filldraw[black] (\x:1cm) circle(0.4pt);
                % draw each angle in degrees
                %\draw (\x:0.6cm) node[fill=white] {$\x^\circ$};
        }

        % draw each angle in radians
        \foreach \x/\xtext in {
            30/\frac{\pi}{6},
            45/\frac{\pi}{4},
            60/\frac{\pi}{3},
            90/\frac{\pi}{2},
            120/\frac{2\pi}{3},
            135/\frac{3\pi}{4},
            150/\frac{5\pi}{6},
            180/\pi,
            210/\frac{7\pi}{6},
            225/\frac{5\pi}{4},
            240/\frac{4\pi}{3},
            270/\frac{3\pi}{2},
            300/\frac{5\pi}{3},
            315/\frac{7\pi}{4},
            330/\frac{11\pi}{6},
            360/2\pi}
                \draw (\x:0.85cm) node[fill=white] {$\xtext$};

        % draw the horizontal and vertical coordinates
        % the placement is better this way
        \draw (-1.25cm,0cm) node[above=1pt] {$(-1,0)$}
              (1.25cm,0cm)  node[above=1pt] {$(1,0)$}
              (0cm,-1.25cm) node[fill=white] {$(0,-i)$}
              (0cm,1.25cm)  node[fill=white] {$(0,i)$};
    \end{tikzpicture}
    	\end{enumerate}
	\item[\textbf{1.}]
	\end{enumerate}
\end{document}
