\newcommand{\authorinfotitle}{Vanessa Closius, Jonas Tietz, Tronje Krabbe}
\newcommand{\authorinfo}{Vanessa Closius, Jonas Tietz, Tronje Krabbe}
\newcommand{\titleinfo}{MMS}
\newcommand{\qed}{\square}

\documentclass[a4paper,11pt]{article}
%\usepackage[german,ngerman]{babel}
\usepackage[utf8]{inputenc}
\usepackage[T1]{fontenc}
\usepackage{lmodern}
\usepackage{amssymb}
\usepackage{mathtools}
\usepackage{amsmath}
\usepackage{enumerate}
\usepackage{breqn}
\usepackage{fancyhdr}
\usepackage{multicol}
\usepackage{tikz}

\author{\authorinfotitle}
\title{\titleinfo}
\date{\today}

\pagestyle{fancy}
\fancyhf{}
\fancyhead[R]{\authorinfo}
\fancyhead[L]{MMS Hausaufgaben}
\fancyfoot[C]{\thepage}
\allowdisplaybreaks
\begin{document}
	\maketitle
	\begin{enumerate}
		% Aufgabe 1
		\item[\textbf{1.}]
		\begin{enumerate}
			\item[\textbf{1)}]
				TODO
			\item[\textbf{2)}]
				TODO
		\end{enumerate}
		% Aufgabe 2
		\item[\textbf{2)}]
		\begin{enumerate}
			\item[\textbf{1)}]
				Faltet man das Rechteckssignal mit sich selbst (also ist $n=1$),
				erhält man ein Dreieckssignal.
				Geht $n$ gegen $\infty$, konvergiert unser Ergebnis zu einem
				Dirac-Stoß. Tatsächlich tritt bei $n=8$ schon ein Fehler auf,
				und unsere Visualisierung scheitert. In dem Fall erreicht
				$f(0)$ so einen hohen Wert, dass \texttt{pyplot} nicht mehr
				damit umgehen kann.
			\item[\textbf{2)}]
				Das Faltungstheorem besagt, dass:
				\begin{align*}
					f * g = F^{-1}\left \{ F\{f\} \cdot F\{g\}\right \}
				\end{align*}
				wobei $f * g$ die Faltung zweier Funktionen $f$ und $g$ ist.
				$F^{-1}$ sei die inverse Fouriertransformation, und $F$ die
				Fouriertransformation.
				Wir haben die linke und die rechte Seite der Gleichung in
				Python implementiert, und zeigen durch das Zeichnen zweier Graphen,
				dass die Gleichung stimmt - denn die Graphen überlappen exakt.
    	\end{enumerate}
	\end{enumerate}
\end{document}
