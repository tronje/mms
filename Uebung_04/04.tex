\newcommand{\authorinfotitle}{Vanessa Closius, Jonas Tietz, Tronje Krabbe}
\newcommand{\authorinfo}{Vanessa Closius, Jonas Tietz, Tronje Krabbe}
\newcommand{\titleinfo}{MMS}
\newcommand{\qed}{\square}

\documentclass[a4paper,11pt]{article}
%\usepackage[german,ngerman]{babel}
\usepackage[utf8]{inputenc}
\usepackage[T1]{fontenc}
\usepackage{lmodern}
\usepackage{amssymb}
\usepackage{mathtools}
\usepackage{amsmath}
\usepackage{enumerate}
\usepackage{breqn}
\usepackage{fancyhdr}
\usepackage{multicol}
\usepackage{tikz}

\author{\authorinfotitle}
\title{\titleinfo}
\date{\today}

\pagestyle{fancy}
\fancyhf{}
\fancyhead[R]{\authorinfo}
\fancyhead[L]{MMS Hausaufgaben}
\fancyfoot[C]{\thepage}
\allowdisplaybreaks
\begin{document}
	\maketitle
	\begin{enumerate}
		% Aufgabe 1
		\item[\textbf{1.}]
		\begin{enumerate}
			\item[\textbf{1)}]
				\begin{enumerate}
				\item[\textbf{a)}]
				\item[\textbf{b)} $f(t)=\cos(t)$]
				\begin{align*}
					c_m&=\frac{1}{2\pi}\int_{-\pi}^{\pi}\cos(t)e^{-jm\omega_0t}dt\\
					   &=\frac{1}{2\pi}\int_{-\pi}^{\pi}\frac{1}{2}(e^{jt}+e^{-jt})e^{-jm\omega_0t}dt\\
					   &=\frac{1}{4\pi}\int_{-\pi}^{\pi}e^{jt-jm\omega_0t}+e^{-jt-jm\omega_0t}dt\\
					   &=\frac{1}{4\pi}\int_{-\pi}^{\pi}e^{jt(1-m\omega_0)}dt+\int_{-\pi}^{\pi}e^{-jt(1+m\omega_0)}dt\\
					   &=\frac{1}{4\pi}\left[\frac{1}{j(1-m\omega_0)} e^{jt(1-m\omega_0)}\right]_{-\pi}^{\pi}+\left[\frac{1}{-j(1+m\omega_0)}e^{-jt(1+m\omega_0)}\right]_{-\pi}^{\pi}\\
					   &=\frac{1}{4\pi}(\frac{1}{j(1-m\omega_0)}(e^{j\pi(1-m\omega_0)}-e^{-j\pi(1-m\omega_0)})\\&+\frac{1}{-j(1+m\omega_0)}(e^{-jpi(1+m\omega_0)}-e^{jpi(1+m\omega_0)}))\\
					   &=\frac{1}{4\pi}(\frac{1}{j(1-m\omega_0)}\frac{2}{-j}\sin(\pi(1-m\omega_0))+\frac{1}{-j(1+m\omega_0)}\frac{2}{-j}\sin(\pi(-1-m\omega_0)))\\
					   &=\frac{1}{4\pi}(\frac{2}{1-m\omega_0}\sin(\pi-\pi m\omega_0))+\frac{2}{-1-m\omega_0}\sin(-\pi-\pi m\omega_0)))\\
					   &=\frac{2}{4\pi-4\pi m\omega_0}\sin(\pi-\pi m\omega_0))+\frac{2}{-4\pi-4\pi m\omega_0}\sin(-\pi-\pi m\omega_0))\\
					   &=\frac{-1}{2\pi-2\pi m\omega_0}\sin(-\pi m\omega_0))+\frac{-1}{-2\pi-2\pi m\omega_0}\sin(-\pi m\omega_0))\\
					   &=\frac{1}{2\pi-2\pi m\omega_0}\sin(\pi m\omega_0))+\frac{1}{-2\pi-2\pi m\omega_0}\sin(\pi m\omega_0))\\
					   &=\sin(\pi m\omega_0)(\frac{1}{2\pi-2\pi m\omega_0}+\frac{1}{-2\pi-2\pi m\omega_0})\\
					   &=\frac{1}{2\pi}\sin(\pi m\omega_0)(\frac{1}{1- m\omega_0}+\frac{1}{-1-m\omega_0})\\
					   &=\frac{1}{2\pi}\sin(\pi m\omega_0)\frac{(-1-m\omega_0)+(1- m\omega_0)}{(1- m\omega_0)(-1-m\omega_0)}\\
					   &=\frac{1}{2\pi}\sin(\pi m\omega_0)\frac{-2m\omega_0}{(-1-m\omega_0+m\omega_0+m^2\omega_0^2)}\\
					   &=\frac{1}{2\pi}\sin(\pi m\omega_0)\frac{-2m\omega_0}{(m^2\omega_0^2-1)}\\
					   &=\frac{-m\omega_0}{\pi m^2\omega_0^2-\pi}\sin(\pi m\omega_0)\\
				\end{align*}
				\end{enumerate}
			\item[\textbf{2)}]
				TODO
			\item[\textbf{3)}]
			\begin{align*}
			g(t) &= rect(t) * rect(t) = \int_{- \infty}^{\infty}rect(\tau - t)\cdot rect(t)dt
			\end{align*}
			wobei $rect(t)$ der Rechteckimpuls ist:
			\begin{align*}
			rect(t) = 
			\begin{cases}
			1 \text{, für } |t| \leq \frac{1}{2} \\
			0 \text{, für } |t| \geq \frac{1}{2}
			\end{cases}
			\end{align*}
			Für die Faltung unterscheiden wir 2 Fälle. \\ \\
			Fall 1: $|\tau| \le 1$ \\
			\begin{align*}
			rect(\tau - t)\cdot rect(t) = 0 \\
			\Rightarrow g(t) = \int_{- \infty}^{\infty} 0 dt = 0 , \tau \not\in [-1, 1]
			\end{align*}
			Fall 2: $|\tau| \ge 1$ \\
			Die Rechtecke überlappen sich, der Überlappungsbereich hat die Breite $\Delta t = 1 - \tau$. Es gilt:
			\begin{align*}
			y(t) &= \int_{- \infty}^{\infty}rect(\tau - t)\cdot rect(t)dt = \int_{\Delta t} 1^2 = \Delta t \cdot 1 \\
			&= \begin{cases}
			1 + \tau &, \tau\in [-1, 0] \\
			1 - \tau &, \tau\in [0,1]
			\end{cases}
			\end{align*}
			Also gilt insgesamt:
			\begin{align*}
			y(t) =
			\begin{cases}
			0 &, \tau\not\in [-1,1] \\
			1 + t &, \tau\in [-1, 0] \\
			1 - t &, \tau\in [0,1]
			\end{cases}
			\end{align*}
			Dies entspricht dem Dreiecksimpuls.
		\end{enumerate}
		% Aufgabe 2
		\item[\textbf{2)}]
		\begin{enumerate}
			\item[\textbf{1)}]
				Faltet man das Rechteckssignal mit sich selbst (also ist $n=1$),
				erhält man ein Dreieckssignal.
				Geht $n$ gegen $\infty$, konvergiert unser Ergebnis zu einem
				Dirac-Stoß. Tatsächlich tritt bei $n=8$ schon ein Fehler auf,
				und unsere Visualisierung scheitert. In dem Fall erreicht
				$f(0)$ so einen hohen Wert, dass \texttt{pyplot} nicht mehr
				damit umgehen kann.
			\item[\textbf{2)}]
				Das Faltungstheorem besagt, dass:
				\begin{align*}
					f * g = F^{-1}\left \{ F\{f\} \cdot F\{g\}\right \}
				\end{align*}
				wobei $f * g$ die Faltung zweier Funktionen $f$ und $g$ ist.
				$F^{-1}$ sei die inverse Fouriertransformation, und $F$ die
				Fouriertransformation.
				Wir haben die linke und die rechte Seite der Gleichung in
				Python implementiert, und zeigen durch das Zeichnen zweier Graphen,
				dass die Gleichung stimmt - denn die Graphen überlappen exakt.
    	\end{enumerate}
	\end{enumerate}
\end{document}
