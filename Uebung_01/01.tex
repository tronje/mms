\newcommand{\authorinfotitle}{Vanessa Closius, Jonas Tietz, Tronje Krabbe}
\newcommand{\authorinfo}{Vanessa Closius, Jonas Tietz, Tronje Krabbe}
\newcommand{\titleinfo}{MMS 01 30.10.2018}
\newcommand{\qed}{\square}

\documentclass [a4paper,11pt]{article}
\usepackage[german,ngerman]{babel}
\usepackage[utf8]{inputenc}
\usepackage[T1]{fontenc}
\usepackage{lmodern}
\usepackage{amssymb}
\usepackage{mathtools}
\usepackage{amsmath}
\usepackage{enumerate}
\usepackage{breqn}
\usepackage{fancyhdr}
\usepackage{multicol}

\author{\authorinfotitle}
\title{\titleinfo}
\date{\today}

\pagestyle{fancy}
\fancyhf{}
\fancyhead[R]{\authorinfo}
\fancyhead[L]{MMS Hausaufgaben}
\fancyfoot[C]{\thepage}

\begin{document}
\maketitle
    \begin{enumerate}
        % Aufgabe 1
        \item[\textbf{1.}]
            \begin{enumerate}
                \item[a)]
                Frequenz ist, salopp gesagt, ein Maß dafür, wie oft sich etwas wiederholt.
                Die Einheit der Frequenz ist die SI-Einheit \textit{Hertz} (Hz), für die gilt:
                $$1 Hz = s^{-1}$$
                In der Natur und der Wissenschaft wird Frequenz oft im Zusammenhang mit Wellen
                erwähnt bzw. benutzt. Eine Welle startet in einem neutralen Zustand,
                erreicht dann ein Maximum, daraufhin ein Minimum und erreicht dann wieder den
                neutralen Zustand. Die Zeit, die für diese sogenannte Oszillation benötigt wird,
                nennt man Periode ($\lambda$). Die Anzahl der Oszillationen pro Zeit ist die Frequenz.
                Die Frequenz einer Welle ist also $$ f = \frac{1}{\lambda} $$

                Beispiele für die Frequenz sind etwa:

                \begin{itemize}
                    \item Der Tag-Nacht-Wechsel. Alle 24 Stunden beginnt ein neuer Tag, sodass
                        $$ \frac{1}{24h} \approx 10^{-5} Hz $$
                    \item WLAN-Signale. Es gibt $2.5$ GHz und $5.0$ GHz Frequenzbänder. Das heißt,
                        dass die Oszillation der elektromagnetischen Wellen der WLAN-Signale bis zu
                        fünf Milliarden Mal pro Sekunde eintritt.
                    \item Das für Menschen wahrnehmbare Licht liegt im Frequenzbereich $400$ THz
                        und $750$ THz.
                \end{itemize}

                \item[b)]
                Aus der Nähe betrachtet stellt das Bild den bekannten Physiker Albert Einstein dar.
                Aus der Ferne ist die Schauspielerin Marilyn Monroe zu sehen.
                Offenbar stellen die hohen Frequenzen in dem Bild Monroe dar, und die
                niedrigen stellen Einstein dar.
            \end{enumerate}
    \end{enumerate}
\end{document}
