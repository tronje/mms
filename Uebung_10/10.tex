\newcommand{\authorinfotitle}{Vanessa Closius, Jonas Tietz, Tronje Krabbe}
\newcommand{\authorinfo}{Vanessa Closius, Jonas Tietz, Tronje Krabbe}
\newcommand{\titleinfo}{MMS}
\newcommand{\qed}{\square}

\documentclass[a4paper,11pt]{article}
%\usepackage[german,ngerman]{babel}
\usepackage[utf8]{inputenc}
\usepackage[T1]{fontenc}
\usepackage{lmodern}
\usepackage{amssymb}
\usepackage{mathtools}
\usepackage{amsmath}
\usepackage{enumerate}
\usepackage{breqn}
\usepackage{fancyhdr}
\usepackage{multicol}
\usepackage{tikz}
\usepackage{trfsigns}
\usepackage{float}

\author{\authorinfotitle}
\title{\titleinfo}
\date{\today}

\pagestyle{fancy}
\fancyhf{}
\fancyhead[R]{\authorinfo}
\fancyhead[L]{MMS Hausaufgaben}
\fancyfoot[C]{\thepage}
\allowdisplaybreaks
\begin{document}
	\maketitle
	\begin{enumerate}
		% Aufgabe 1
		\item[\textbf{1.}]
		
		\begin{enumerate}
		% 1.a
		\item[\textbf{a)}]
		\begin{align*}
		E\lbrace aX+bY+c\rbrace &= \iint (ax+by+c)p(x,y)dxdy\\
				   &= \iint ax \cdot p(x,y) + by \cdot p(x,y) + c \cdot p(x,y) dxdy\\
				   &= \iint ax \cdot p(x,y) dxdy + \iint by \cdot p(x,y) dxdy + \iint c \cdot p(x,y) dxdy\\
				  &= \int ax \cdot p(x) dx + \int by \cdot p(y) dy +  c\\ 
				  &= a\int x \cdot p(x) dx + b\int y \cdot p(y) dy +  c\\ 
				  &= aE\lbrace X\rbrace + bE\lbrace Y\rbrace +  c
		\end{align*}
		\item[\textbf{b)}]
		\begin{align*}
		E\lbrace (X-\mu_x)^2\rbrace &= E\lbrace X^2-2X\mu_x + \mu^2_x\rbrace \\
							  		&= E\lbrace X^2\rbrace-E\lbrace 2X\mu_x \rbrace + \mu^2_x\\
							  		&= E\lbrace X^2\rbrace-2\mu_x E\lbrace X \rbrace + \mu^2_x\\				
							  		&= E\lbrace X^2\rbrace-2 E\lbrace X \rbrace E\lbrace X \rbrace + E\lbrace X \rbrace^2\\			
							  		&= E\lbrace X^2\rbrace-2 E\lbrace X \rbrace^2 + E\lbrace X \rbrace^2\\				
							  		&= E\lbrace X^2\rbrace- E\lbrace X \rbrace^2 \\					  						&= E\lbrace X^2\rbrace- \mu_x^2
		\end{align*}
		\item[\textbf{c)}]
		\begin{align*}
		V\lbrace aX + b \rbrace &= E\lbrace (aX + b)^2 \rbrace - E^2\lbrace (aX + b) \rbrace \\
		&= E\lbrace (a^2X^2 + 2abX + b^2) \rbrace - E\lbrace (aX + b) \rbrace \cdot
		E\lbrace (aX + b) \rbrace \\
		&= a^2E\lbrace X^2 \rbrace + 2abE\lbrace X \rbrace + b^2 - (aE\lbrace X \rbrace + b) \cdot (aE\lbrace X \rbrace + b) \\
		&= a^2E\lbrace X^2 \rbrace + 2abE\lbrace X \rbrace + b^2 - (a^2E^2\lbrace X \rbrace + 2abE\lbrace X \rbrace + b^2) \\
		&= a^2E\lbrace X^2 \rbrace - a^2E^2\lbrace X \rbrace \\
		&= a^2 \cdot (E\lbrace X^2 \rbrace - 2E^2\lbrace X \rbrace) \\
		&= a^2 \cdot V\lbrace X \rbrace
		\end{align*}
		\item[\textbf{d)}]
		\begin{align*}
		E\lbrace XY \rbrace &= \iint xy \cdot p(x,y) dxdy \\
		&= \iint xy \cdot \underbrace{p(x) \cdot p(y)}_{\text{independence}} dxdy \\
		&= \int x \cdot p(x) dx \cdot \int y \cdot p(y) dy \\
		&= E\lbrace X \rbrace \cdot E\lbrace Y \rbrace
		\end{align*}
		\item[\textbf{e)}]
		\begin{align*}
		V\lbrace aX + bY \rbrace &= E\lbrace (aX + bY)^2 \rbrace - E^2\lbrace (aX + bY) \rbrace \\
		&= E\lbrace (a^2X^2 + 2abXY + b^2Y^2) \rbrace - E\lbrace (aX + bY) \rbrace \cdot
		E\lbrace (aX + bY) \rbrace \\
		&= a^2E\lbrace X^2 \rbrace + 2abE\lbrace XY \rbrace + b^2E\lbrace Y^2 \rbrace - \\ &(aE\lbrace X \rbrace + bE\lbrace Y \rbrace) \cdot (aE\lbrace X \rbrace + bE\lbrace Y \rbrace) \\
		&= a^2E\lbrace X^2 \rbrace + 2abE\lbrace XY \rbrace + b^2E\lbrace Y^2 \rbrace - \\ &(a^2E^2\lbrace X \rbrace + \underbrace{2abE\lbrace XY \rbrace}_{\text{independence}} + b^2E^2\lbrace Y \rbrace) \\
		&= a^2E\lbrace X^2 \rbrace + E\lbrace Y^2 \rbrace - a^2E^2\lbrace X \rbrace - b^2E^2\lbrace Y \rbrace\\
		&= a^2 \cdot (E\lbrace X^2 \rbrace - 2E^2\lbrace X \rbrace) + b^2 \cdot (E\lbrace Y^2 \rbrace - 2E^2\lbrace Y \rbrace) \\
		&= a^2 \cdot V\lbrace X \rbrace + b^2 \cdot V\lbrace Y \rbrace
		\end{align*}
		\end{enumerate}

		\item[\textbf{2.}]
		\begin{enumerate}
			\item[\textbf{a)}]
			We replace the integral's limits with 0 and 1 respectively, because
			our probability density function is 0 outside of those limits.
			\begin{align*}
				\mu_X &= \int_{-\infty}^\infty x_n f_{X_n}(x_n) dx \\
					  &= \int_0^1 x_n 2x_n dx \\
					  &= \int_0^1 2x_n^2 dx \\
					  &= \bigg[ \frac{2}{3} x_n^3 \bigg]_0^1 \\
					  &= \frac{2}{3} - 0 = \frac{2}{3}
			\end{align*}

			\begin{align*}
				\sigma_X^2 &= E(X^2) - \mu_X^2 \\
						   &= E(X^2) - \frac{2}{3}^2 \\
						   &= \int_0^1 x_n^2 2x_n dx - \frac{4}{9} \\
						   &= \int_0^1 2x_n^3 dx - \frac{4}{9} \\
						   &= \bigg[ \frac{2}{4} x_n^4 \bigg]_0^1 - \frac{4}{9} \\
						   &= \frac{1}{2} - \frac{4}{9} = \frac{1}{18}
			\end{align*}

			\item[\textbf{b)}]
			\begin{align*}
				E(Y_n) &= E(X_n + X_{n - 1}) = E(X_n) + E(X_{n + 1}) \\
					   &= \int\int (x_n + x_{n + 1}) f_{X_n}(x_n) f_{X_{n + 1}}(x_{n + 1}) dx_n dx_{n + 1} \\
					   &= \int x_n f_{X_n}(x_n) dx_n + \int x_{n + 1} f_{X_{n + 1}}(x_{n + 1}) dx_{n + 1} \\
					   &= \frac{2}{3} + \frac{2}{3} = \frac{4}{3}
			\end{align*}
		\end{enumerate}
	\end{enumerate}
\end{document}
