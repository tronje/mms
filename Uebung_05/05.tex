\newcommand{\authorinfotitle}{Vanessa Closius, Jonas Tietz, Tronje Krabbe}
\newcommand{\authorinfo}{Vanessa Closius, Jonas Tietz, Tronje Krabbe}
\newcommand{\titleinfo}{MMS}
\newcommand{\qed}{\square}

\documentclass[a4paper,11pt]{article}
%\usepackage[german,ngerman]{babel}
\usepackage[utf8]{inputenc}
\usepackage[T1]{fontenc}
\usepackage{lmodern}
\usepackage{amssymb}
\usepackage{mathtools}
\usepackage{amsmath}
\usepackage{enumerate}
\usepackage{breqn}
\usepackage{fancyhdr}
\usepackage{multicol}
\usepackage{tikz}

\author{\authorinfotitle}
\title{\titleinfo}
\date{\today}

\pagestyle{fancy}
\fancyhf{}
\fancyhead[R]{\authorinfo}
\fancyhead[L]{MMS Hausaufgaben}
\fancyfoot[C]{\thepage}
\allowdisplaybreaks
\begin{document}
	\maketitle
	\begin{enumerate}
		% Aufgabe 1
		\item[\textbf{1.}]
		% Aufgabe 2
		\item[\textbf{2)}]
			\begin{enumerate}
				\item[a)]
					Stimmen die beide Sinus-Funktionen überein, so ist der
					Korrelationzkoeffizient 1 --- denn die Funktionen sind ja
					identisch. Um $\pi$ verschoben, müsste der Koeffizient -1
					sein, da
					\begin{align*}
					\sin(x + \pi) = -\sin(x)
					\end{align*}
					Die Funktionen stimmen sozusagen eigentlich überein,
					bis auf das Vorzeichen der einzelnen Werte.

					Der Koeffizient ist (praktisch) 0 bei einer Verschiebung
					von $\frac{\pi}{2}$.
			\end{enumerate}
	\end{enumerate}
\end{document}
