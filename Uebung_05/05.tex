\newcommand{\authorinfotitle}{Vanessa Closius, Jonas Tietz, Tronje Krabbe}
\newcommand{\authorinfo}{Vanessa Closius, Jonas Tietz, Tronje Krabbe}
\newcommand{\titleinfo}{MMS}
\newcommand{\qed}{\square}

\documentclass[a4paper,11pt]{article}
%\usepackage[german,ngerman]{babel}
\usepackage[utf8]{inputenc}
\usepackage[T1]{fontenc}
\usepackage{lmodern}
\usepackage{amssymb}
\usepackage{mathtools}
\usepackage{amsmath}
\usepackage{enumerate}
\usepackage{breqn}
\usepackage{fancyhdr}
\usepackage{multicol}
\usepackage{tikz}
\usepackage{trfsigns}

\author{\authorinfotitle}
\title{\titleinfo}
\date{\today}

\pagestyle{fancy}
\fancyhf{}
\fancyhead[R]{\authorinfo}
\fancyhead[L]{MMS Hausaufgaben}
\fancyfoot[C]{\thepage}
\allowdisplaybreaks
\begin{document}
	\maketitle
	\begin{enumerate}
		% Aufgabe 1
		\item[\textbf{1.}]
			\begin{enumerate}
				\item[a)]
				Da die Ableitung von Sinus Cosinus ist, wie man sich auch leicht am Einhaltskreis veranschaulichen kann, gilt:
				\begin{align*}
				\cos(\Omega t) = \frac{d}{dt} \sin(\Omega t) &\laplace (j\omega) \cdot (\frac{1}{2}j \cdot (\delta(\omega + \Omega) - \delta(\omega - \Omega)) \\
				&= j^2\omega\frac{1}{2} \cdot (\delta(\omega + \Omega) - \delta(\omega - \Omega)) \\
				&= \frac{1}{2} \cdot (\delta(\omega + \Omega) \cdot \underbrace{(-\omega)}_{\omega = - \Omega} - \delta(\omega - \Omega) \cdot  \underbrace{(-\omega)}_{\omega = \Omega})
				\end{align*}
				Und für $\Omega = 1$ folgt:
				\begin{align*}
				\cos(\Omega t) = \frac{d}{dt} \sin(\Omega t) &\laplace \frac{1}{2} \cdot (\delta(\omega + \Omega) + \delta(\omega - \Omega))
				\end{align*}
				\item[b)]
					Sei
					\begin{align*}
						s(t) = A_1 \cos(t) \\
						g(t) = A_2 \cos(t)
					\end{align*}
					und gelte
					\begin{align}
						A_1 \neq A_2
					\end{align}
					Wir berechnen die Korrelation von $s(t)$ und $g(t)$:
					\begin{align*}
						\rho_{sg} &= \frac{\displaystyle \lim_{T \to \infty} \frac{1}{2T} \int_{-T}^T A_1 A_2 \cos^2(t) dt}{\sqrt{\rho_s\rho_g}} \\
								  &= \frac{\displaystyle \lim_{T \to \infty} \frac{A_1A_2}{2T} \int_{-T}^T \cos^2(t) dt}{\sqrt{\rho_s\rho_g}}
					\end{align*}
			\end{enumerate}

		% Aufgabe 2
		\item[\textbf{2)}]
			\begin{enumerate}
				\item[a)]
					Stimmen die beiden Sinus-Funktionen überein, so ist der
					Korrelationzkoeffizient 1 --- denn die Funktionen sind ja
					identisch. Um $\pi$ verschoben, müsste der Koeffizient -1
					sein, da
					\begin{align*}
					\sin(x + \pi) = -\sin(x)
					\end{align*}
					Die Funktionen stimmen sozusagen eigentlich überein,
					bis auf das Vorzeichen der einzelnen Werte.

					Der Koeffizient ist (praktisch) 0 bei einer Verschiebung
					von $\frac{\pi}{2}$.
			\end{enumerate}
	\end{enumerate}
\end{document}
